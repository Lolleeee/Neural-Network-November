\documentclass[11pt]{article}
\usepackage[utf8]{inputenc}
\usepackage[english]{babel}
\usepackage{amsmath}
\usepackage{graphicx}
\usepackage{float}
\usepackage{lipsum}
\usepackage{multicol}
\usepackage{xcolor}
\usepackage{tabularx}
\usepackage{booktabs}
\usepackage{hyperref}
\newcolumntype{Y}{>{\centering\arraybackslash}X}
\usepackage[left=2.00cm, right=2.00cm, top=2.00cm, bottom=2.00cm]{geometry}

\title{AN2DL Reports Template}

\begin{document}
    
    \begin{figure}[H]
        \raggedright
        \includegraphics[scale=0.4]{reports/images/polimi.png} \hfill \includegraphics[scale=0.3]{reports/images/airlab.jpeg}
    \end{figure}
    
    \vspace{5mm}
    
    \begin{center}
        % Select between First and Second
        {\Large \textbf{AN2DL - First/Second Homework Report}}\\
        \vspace{2mm}
        % Change with your Team Name
        {\Large \textbf{Team Name}}\\
        \vspace{2mm}
        % Team Members Information
        {\large Michael Alibeaj,}
        {\large Matteo Bettiati,}
        {\large Lorenzo Bianchi,}
        {\large Francesco Ostidich}\\
        \vspace{2mm}
        % Codabench Nicknames
        {michaelgear01,}
        {betti38,}
        {lollyx21,}
        {francescoostidich}\\
        \vspace{2mm}
        % Matriculation Numbers
        {Matricola1,}
        {258730,}
        {Matricola3,}
        {Matricola4}\\
        \vspace{5mm}
        \today
    \end{center}    
    \vspace{5mm}
    
    \begin{multicols}{2}
        
        \section{Introduction}
        In this assignment, we were provided with 64x128 grayscale images of Mars terrain, where each pixel is categorized into one of five terrain classes. This is a \textit{semantic segmentation problem}, and our objective is to \textbf{assign the correct class label} to each pixel in the image. To address this challenge, we analyzed the data, built a UNet model, and enhanced its architecture to improve segmentation accuracy.

        
        \section{Problem Analysis}
        \subsection{Dataset Details}
        
        The dataset consists of 64x128 grayscale images (1 channel), stored in \texttt{npz} (Numpy archive) format. There are five classes: 0 for Background, 1 for Soil, 2 for Bedrock, 3 for Sand, and 4 for Big Rock. The training data is stored in the \texttt{mars\_for\_students.npz} file, which contains two fields: \texttt{"training\_set"}, a Numpy array of shape (2615, 2, 64, 128) with image and label arrays, and \texttt{"test\_set"}, a Numpy array of shape (10022, 64, 128) containing unlabeled test images.
        
        \subsection{Main Challenges}
        
        Several challenges arise in this task: the small image size limits detail, and varying terrain types can blur class boundaries. Class imbalance, with Background dominating, may hurt performance on underrepresented classes like Big Rock. The Mars terrain’s complexity and irregular textures further complicate model generalization.
        
        \section{Methods}
        
        To improve model performance, we employed several strategies:
        
        \subsection{Data Augmentation}
        To increase dataset diversity and improve generalization, we applied random rotations, flips, and scaling to the images. These augmentations helped the model handle variations in terrain orientation and spatial configuration.
        
        \subsection{UNet Architecture}
        We built a custom UNet model from scratch, utilizing an encoder-decoder structure with skip connections to preserve spatial information. This architecture enabled the model to extract hierarchical features and make pixel-wise predictions. We trained the model on the augmented dataset to map images to their respective terrain labels.
        
        \subsection{Loss Function Exploration}
        We experimented with various loss functions. We started with categorical cross-entropy and later explored mean Dice and Intersection over Union (mIoU) losses to handle class imbalances. These helped the model focus on smaller classes like Big Rock and improve both global and local segmentation accuracy.
        
        \section{Experiments}
        
        We tested different configurations to optimize the UNet model for the Mars terrain segmentation task.
        
        \subsection{Loss Functions}
        We used a combined loss of categorical cross-entropy and Dice coefficient to balance class accuracy and segmentation overlap. We also tested Dice loss and sparse categorical cross-entropy separately, focusing on class balance and overall performance.
        
        \subsection{Model Architecture Adjustments}
        We fine-tuned the UNet layers, adjusting the number of convolutional layers, filter sizes, and kernel depths to improve feature extraction and pixel-wise prediction. These adjustments helped the model capture more complex terrain features.
        
        \subsection{Advanced Architectures}
        We explored advanced architectures such as dual UNet, which combines outputs from two parallel UNet models, and Transformers, which capture long-range dependencies. Additionally, we incorporated residual blocks to improve gradient flow and prevent overfitting, enhancing the model’s learning capacity.





        \section{Results}
        Present your main findings here. You might want to:
        \begin{itemize}
            \item Compare your results with baselines
            \item Highlight key achievements using \textbf{bold text}
            \item Explain any unexpected outcomes
        \end{itemize}

        \section{Discussion}
        In this section, analyse your results critically. Consider:
        \begin{itemize}
            \item Strengths and weaknesses
            \item Limitations and assumptions
        \end{itemize}

        \section{Conclusions}
        \subsection{Contribution}

        Each team member contributed equally to the project, with specific responsibilities as follows. \textbf{Alibeaj} worked on the development and implementation of the UNet architecture. \textbf{Bettiati} focused on refining and optimizing the UNet model for better performance. \textbf{Bianchi} investigated and experimented with various loss functions to improve segmentation accuracy. \textbf{Ostidich} led the data augmentation efforts, applying various transformations to enhance model generalization.

        \subsection{References}

    
    \end{multicols}
\end{document}